\documentclass[oneside,11pt]{memoir}

% Fonts
\usepackage[T1]{fontenc}    % better output encoding
\usepackage{textcomp}       % For ASCII diacritics
\usepackage{lmodern}        % Latin Modern instad of CM
%\usepackage{inconsolata}   % better monospaced font
\usepackage[defaultsans]{droidsans} % straight quotes
\usepackage{microtype}      % fix the little things % comes after any fonts
\usepackage{multicol}

% basics
\usepackage[textwidth=0.75\paperwidth]{geometry}    % Adjust margins
\usepackage{url}        % makes urls look nicer
\usepackage[usenames,dvipsnames]{color} % adds colored text
\usepackage[font=footnotesize]{caption} % more control over figure captions
\usepackage{underscore} % better text underscores
\usepackage{tabularx}   % more flexible tables
\usepackage[table,dvipsnames]{xcolor}   % alternating table colors
\usepackage{todonotes}  % backend organization. Also for some reason necessary for tikz
\usepackage{booktabs}
\usepackage{ltxtable}
\usepackage{array}
\usepackage{upquote}
\usepackage{tabu}
\usepackage[level]{datetime}
\usepackage{listings}
\usepackage{enumitem}   % better description environment
%\shortdate
%\newdateformat{mydate}{%
%   \monthname[\THEMONTH] \THEDAY, \THEYEAR}

% Lists
\usepackage[ampersand]{easylist}    % Nicer syntax for creating lists
\usepackage{paralist}   % Allows inline lists %
% For citations using Bibtex
\usepackage{natbib}

% Hyperref
% allows hyperlinking % NEEDS TO BE LOADED LAST
\usepackage[linkcolor=magenta,colorlinks=true,
% draft,
breaklinks=true]{hyperref}
\newcommand{\sdmf}{\textsc{sdmf}}

\renewcommand{\thesection}{\arabic{section}}
%\renewcommand{\thesubsection}{\arabic{section}.\arabic{subsection}}

\setsecnumdepth{subsection}

\title{Makefile Documentation}

\date{\today}
\author{}

\usepackage{titletoc}
\titlecontents*{section} % Section
    [0em]                  % Left
    {\space}               % Above code
    {\thecontentslabel~}   % Numbered format
    {}                     % Numberless format
    {}                     % Filler
    [\hspace{1em}]         % Separator

\makeatletter
\renewcommand*{\@tocmaketitle}{\noindent \textbf{\large Contents}}
\makeatother

\definecolor{lemon}{HTML}{F5ECCE}

\newcommand{\Today}{\usdate\today}

% \rewnewcommand{\arraystretch}{1.5}

\newcommand{\ExternalLink}{%
    \tikz[x=1.2ex, y=1.2ex, baseline=-0.05ex]{%
        \begin{scope}[x=1ex, y=1ex]
            \clip (-0.1,-0.1)
                --++ (-0, 1.2)
                --++ (0.6, 0)
                --++ (0, -0.6)
                --++ (0.6, 0)
                --++ (0, -1);
            \path[draw,
                line width = 0.5,
                rounded corners=0.5]
                (0,0) rectangle (1,1);
        \end{scope}
        \path[draw, line width = 0.5] (0.5, 0.5)
            -- (1, 1);
        \path[draw, line width = 0.5] (0.6, 1)
            -- (1, 1) -- (1, 0.6);
        }
    }

\begin{document}

    \input{descriptions.tex}

    \begin{center}
        \fbox{Note that the items are sorted uppercase, then lowercase:
                [A-Za-z]}
    \end{center}

    \section{Targets}
    \label{targets}

%   \LTXtable{\textwidth}{targets.tex}
    \tabulinesep=1.2mm
    \rowcolors{2}{gray!25}{white}
    \begin{longtabu} to \linewidth {>{\ttfamily}p{5.4cm} X }
        \rowcolor{gray!50}
        \textbf{Target} & \textbf{Description} \\

        \rowcolors{2}{gray!25}{white}

\begin{longtable}{>{\ttfamily}l X >{\ttfamily}l}
	\rowcolor{gray!50}
	\textbf{Target} & \textbf{Definition \& Description} & \textbf{File} \\
	
	\rowcolors{2}{gray!25}{white}

\begin{longtable}{>{\ttfamily}l X >{\ttfamily}l}
	\rowcolor{gray!50}
	\textbf{Target} & \textbf{Definition \& Description} & \textbf{File} \\
	
	\rowcolors{2}{gray!25}{white}

\begin{longtable}{>{\ttfamily}l X >{\ttfamily}l}
	\rowcolor{gray!50}
	\textbf{Target} & \textbf{Definition \& Description} & \textbf{File} \\
	
	\input{../sdmf-output/targets.txt}		
	
	\bottomrule
\end{longtable}
		
	
	\bottomrule
\end{longtable}
		
	
	\bottomrule
\end{longtable}


    \end{longtabu}

    \section{Variables}
    \label{variables}

    \textbf{Note}: Variables with an asterisk (``\texttt{*}'') are global
        variables initialized with ``\texttt{export}.''
        Read more about exporting variables in the GNU Make manual:
        5.7.2 Communicating Variables to a Sub-\texttt{make}
        \href{https://www.gnu.org/software/make/manual/html_node/%
                Variables_002fRecursion.html}{\ExternalLink}.

    % Symbol: https://tex.stackexchange.com/q/99316

    \rowcolors{2}{gray!25}{white}

    \begin{longtabu} to \linewidth {>{\ttfamily}p{5.4cm} X}
        \rowcolor{gray!50}
        \textbf{Variable} & \textbf{Definition \& Description} \\

        \rowcolors{2}{gray!25}{white}

\begin{longtable}{>{\ttfamily}l X >{\ttfamily}l}
	\rowcolor{gray!50}
	\textbf{Variable} & \textbf{Definition \& Description} & \textbf{File} \\
	
	\rowcolors{2}{gray!25}{white}

\begin{longtable}{>{\ttfamily}l X >{\ttfamily}l}
	\rowcolor{gray!50}
	\textbf{Variable} & \textbf{Definition \& Description} & \textbf{File} \\
	
	\rowcolors{2}{gray!25}{white}

\begin{longtable}{>{\ttfamily}l X >{\ttfamily}l}
	\rowcolor{gray!50}
	\textbf{Variable} & \textbf{Definition \& Description} & \textbf{File} \\
	
	\input{../sdmf-output/variables.txt}		
		
	\bottomrule
\end{longtable}		
		
	\bottomrule
\end{longtable}		
		
	\bottomrule
\end{longtable}

        \bottomrule
    \end{longtabu}

    \section{Intermediate Files}
    \label{intermediates}

    \begin{description}
        \input{intermediates.txt}
    \end{description}

    \newpage
    \newgeometry{scale=0.85, centering}
    \section{Makefiles}
    \label{makefiles}

    \lstset{language=make,
      backgroundcolor=\color{lemon},
      showstringspaces=false,
      gobble=8,
      basicstyle=\small\ttfamily,
      breaklines=true,
      escapeinside={\%*}{*},
      upquote=true}

    \subsection{\texttt{Flex.mk}}
\label{Flex.mk}
\lstinputlisting[style=makefile]{../Flex.mk}


\end{document}