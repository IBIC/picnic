\documentclass[oneside,11pt]{article}

% Fonts
\usepackage[T1]{fontenc}	% better output encoding
\usepackage{textcomp}		% For ASCII diacritics
\usepackage{lmodern}		% Latin Modern instad of CM
%\usepackage{inconsolata}	% better monospaced font
\usepackage[defaultsans]{droidsans} % straight quotes
\usepackage{microtype}		% fix the little things % comes after any fonts

% basics
\usepackage{geometry}	% Adjust margins
\usepackage{graphicx}	% now you can add pretty pictures
\usepackage{url} 		% makes urls look nicer
\usepackage[usenames,dvipsnames]{color} % adds colored text
\usepackage[font=footnotesize]{caption}	% more control over figure captions
\usepackage{xparse}		% nesting arguments within environments
\usepackage{underscore}	% better text underscores
\usepackage{marginnote}	% better margin notes
\usepackage{tabularx}	% more flexible tables
\usepackage[table,dvipsnames]{xcolor}	% alternating table colors
\usepackage{todonotes}	% backend organization. Also for some reason necessary for tikz
\usepackage{authblk} % author block
\usepackage{booktabs}
\usepackage{ltxtable}
\usepackage{upquote}
%\usepackage[level]{datetime}
%\shortdate


% Complex packages
\usepackage{tikz}		% nice flowcharts and stuff
\usepackage{tikzpagenodes}
\usetikzlibrary{positioning,calc}
\usepackage{tcolorbox} 	% frames
\usepackage{listings} 	% nice code formatting
\usepackage{dirtree}	% pretty directory trees
\usepackage{minted}

% Lists
\usepackage[ampersand]{easylist}	% Nicer syntax for creating lists
\usepackage{paralist}	% Allows inline lists %
% For citations using Bibtex
\usepackage{natbib}

% Hyperref
\usepackage[colorlinks=true]{hyperref} 	% allows hyperlinking % NEEDS TO BE LOADED LAST

\newcommand{\sdmf}{\textsc{sdmf}}


\newcommand{\bashcmd}[1]{ \hfill\, \begin{minipage}[t]{\linewidth}  \hrule \vspace{0.5\baselineskip} \texttt{\small \$ #1} \vspace{0.5\baselineskip} \hrule \end{minipage} \vspace{0.5\baselineskip} }

\title{Self-Documenting Makefiles!}
\author{Trevor K. McAllister-Day}
\affil{\url{tkmday@uw.edu}}

\usepackage{titletoc}
\titlecontents*{section} % Section
[0em]                  % Left
{\space}               % Above code
{\thecontentslabel~}   % Numbered format
{}                     % Numberless format
{}                     % Filler
[\hspace{1em}]         % Separator


\begin{document}
	
	\maketitle
	
%	\tableofcontents*
	
	\begin{abstract}
		The purpose of this document is to describe the standards for creating self-documenting makefiles (\sdmf). As projects grow more complex and branch off into multiple makefiles, keeping track of variable-settings and targets becomes increasingly difficult. Using \sdmf{} allows you to compile the locations of the variable, targets, and files accessible to all your makefiles into a single place that is searchable; reducing possible collisions between variable and target names.
	\end{abstract}
	
	\section{Introduction}
	
	\subsection{Purpose}
	
	This script is designed to collect the following information from any number of relevant makefiles you point it at:
	
	\begin{easylist}[itemize]
		& \textbf{Top-level descriptions:} ``Header information'' usually found at the top of a makefile, descriptions like
			\begin{lstlisting}[basicstyle=\ttfamily, gobble=24, breaklines=true]
				#* Top level makefile
				#* make all will recursively make specified targets in each subject directory.
			\end{lstlisting}
		& \textbf{Variables:} Defined through \texttt{VAR=x} syntax in the makefile preamble. Often references to directory paths (\texttt{bin/}, \texttt{incoming/}, etc), sometimes capture the output of \texttt{\$(\ldots)} command expansion.
		& \textbf{Targets:} What's actually called from the command line, things like \texttt{PrepSubject}. These are collected from the definition of \texttt{.PHONY}. 
			\begin{center}
				\fbox{\color{red} Targets not set in \texttt{.PHONY} will be considered files, not targets.}
			\end{center}
		& \textbf{Files:} Intermediate targets not called directly from the command line, something like \texttt{mprage/T1\_brain.nii.gz}.
	\end{easylist}
	
	Additionally, Picnic recognizes directives such as \texttt{\#*NODOC} which skips the entire makefile and \texttt{\#*SKIP} which skips the relevant target.
	
	\subsection{Organization}
	
	I've improved the directory structure for picnic; currently it only requires three files:
	
	\begin{easylist}
		& \texttt{picnic}
		& \texttt{makemakedoc.py}
		& \texttt{tables.tex}
	\end{easylist}
	
	These files can be moved to a directory like \texttt{\textasciitilde/local/picnic}, which you can add to your path, allowing you to call Picnic from everywhere.
	
	
	\section{Calling the script}
	\label{sec:calling}
	
	There are two primary ways to document your makefiles.
	
	You can give it explicit direction to the makefiles you would like to document:	
	\bashcmd{picnic PrepSubject.mk subdir/anotherfile.mk}
	
	Or, you can give it a directory from which it will collect all makefiles (in that directory and all subdirectories):	
	\bashcmd{picnic -D lib/makefiles}
	
	Using the \texttt{-D} syntax will include any makefile that ends in \texttt{*.mk} or matches the regex \texttt{*[Mm]akefile*}.
	
	\subsection{Skipping some files}
	
	Because it's possible you would like to not document all makefiles in the directory you have given (for example, a reference makefile), there are two ways to exclude some makefiles:
	
	\begin{easylist}
		& The option \texttt{-e}, to which you can pass a regex. All files matching said regex \textit{will not} be documented. 
		& It is also possible to include a directive in the makefile itself which will cause the documenter to skip it entirely: \texttt{\#*NODOC}. Picnic will check the entire document for \texttt{\#*NODOC}, and if it finds it anywhere, the makefile will be skipped.
	\end{easylist}
	
	\subsection{Options}
	
	Aside from the flags \texttt{-D} and \texttt{-e} discussed above, there is also the standard \texttt{-v} flag (for verbose), which mostly stops silencing the output of PDF\LaTeX, and \texttt{-h}, which outputs the standard help message. \texttt{-e} is not compatible with the first (non-\texttt{D}) syntax.
	
	There is also the option \texttt{-o}, which allows you to specify the name of the output file. If it does not already end in \texttt{.pdf}, it will be \texttt{basename}'d and \texttt{.pdf} will be added.
	
	{\color{red} All options must come before the file list or the directory.} If you place options after the file list or directory, Picnic will hang indefinitely. This is a high-priority bug. In the mean time, make sure to call picnic like:	
	\bashcmd{picnic -o foo.pdf foo.mk}
	
	\textit{not}	
	\bashcmd{picnic foo.mk -o foo.pdf}
	
	\begin{minipage}{\textwidth}
		\textbf{Summary table:}	\\
		\begin{tabularx}{\textwidth}{>{\ttfamily}l X}
			-D & Document all the makefiles in this directory. \\
			-e & Exclude all files matching this regex. (\texttt{-D} only) \\
			-o & Specify the output filename. \\
			-v & Run in verbose mode. \\
			-h & Display help and exit.  \\
		\end{tabularx}
	\end{minipage}
		
	\subsection{Naming makefiles}
	
	In general, \LaTeX{} doesn't like filenames with underscores in them. Picnic will do its best to handle them, but I recommend avoiding underscores where possible. 
	
	\section{Makefile commenting syntax}
	
	Because we have four things to identify from the makefiles (description/directives, variable, targets, and files), I introduce four comment ``keywords'' to facilitate information extraction, and for ease of use with other utilities such as \texttt{grep}.
	
	All makefile comments begin with \texttt{\#}, so \sdmf{} comments will also begin with \texttt{\#}. The four comment styles are: \texttt{\#*}, \texttt{\#!}, \texttt{\#?}, and \texttt{\#>}.
	
	The parser (I believe) can handle most symbols, with \texttt{\&} being replaced by \texttt{and} and semicolons being replaced by commas. Please let me know if any symbols have escaped my notice.
	
	\subsection{\texttt{\#}, \texttt{\#\#}, \texttt{\ldots}}
	
	(Any number of only octothorpes.) \\ Only octothorpes without any \sdmf{} control characters will be ignored, allowing use of  makefile-only comments, section headers and the like. \\ Example: \texttt{\#\#\# QA \#\#\#} will be completely ignored by the parser.
	
	\subsection{\texttt{\#*}}
	
	This comment style is for descriptions at the top of the makefile, or header information. It is also used for directives like \texttt{\#*NODOC} and \texttt{\#*SKIP} (at this time, the only directives).
	
	\subsection{\texttt{\#!}}
	
	These comments explain what a variable is for. The actual value of the variable is captured by the parser, and as such, it is not necessary to include the value itself in the comments. \\
	Example: 
	\begin{lstlisting}[gobble=16, basicstyle=\ttfamily]
		# top level of the project directory
		PROJECT_DIR=/mnt/home/adrc/ADRC 
	\end{lstlisting}
	
	\ldots will result in an entry like: \\
	
	\begin{tabularx}{\textwidth}{>{\ttfamily}l X >{\ttfamily}l}
		\rowcolor{gray!50}
		\textbf{Variable} & \textbf{Definition \& Description} & \textbf{File} \\
		PROJECT\_DIR	& \texttt{/mnt/home/adrc/ADRC} \newline top level of the project directory & foo.mk \\
		\bottomrule
	\end{tabularx}
	
	\subsection{\texttt{\#?}}
	
	These comments are for targets, and should take the form of one-line descriptions of targets to be called from the command line, e.g. \texttt{all}, \texttt{PrepSubject}. \\
	Example:
	\begin{lstlisting}[basicstyle=\ttfamily,gobble=16, breaklines=true]
		#? Make all the relevant PCASL registrations
		PCASL: pcasl/pcasl_fnirt.nii.gz pcasl/pcasl_M0_to_T1_Warped.nii.gz	
	\end{lstlisting} 
	
	Target comments can be multiple-lines (each prefaced with \texttt{\#?}), but they really \textit{shouldn't} be.
	
	\subsection{\texttt{\#>}}

	File (intermediary target) descriptions, any number of \texttt{\#>} comments can be used before an intermediate target. I don't \textit{recommend} using more than two or three. Newlines will not be preserved, and will be replaced with semicolons. \\ 
	Example: 
	\begin{lstlisting}[basicstyle=\ttfamily,gobble=16,breaklines=true]
		#> Register m0 to t1 using fnirt
		pcasl/pcasl_fnirt.nii.gz: pcasl/Pcasl_skstrip.nii.gz mprage/T1_brain.nii.gz
	\end{lstlisting}

\end{document}